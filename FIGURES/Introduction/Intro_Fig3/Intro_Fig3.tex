\documentclass{standalone}
\usepackage{tikz}
\usepackage{tikz-3dplot}
\usetikzlibrary{3d}
\usepackage{parskip}
\usepackage{tkz-berge}
\usetikzlibrary{shapes.geometric}
\usetikzlibrary{shapes}


\begin{document}
	
			\tdplotsetmaincoords{2.71828+45}{2.71828+45}
\begin{tikzpicture}
\begin{scope}[tdplot_main_coords,scale=.1,blue,line width=7pt,line join=round]

\path(-150cm,-150cm)(150cm,150cm);
	\draw
	(20.1774,0,-97.9432)--(40.3548,-32.6477,-85.4729)--(73.0026,-20.1774,-65.2955)--(73.0026,20.1774,-65.2955)--(40.3548,32.6477,-85.4729)--cycle
	(20.1774,0,97.9432)--(40.3548,32.6477,85.4729)--(73.0026,20.1774,65.2955)--(73.0026,-20.1774,65.2955)--(40.3548,-32.6477,85.4729)--cycle
	(-20.1774,0,97.9432)--(-40.3548,-32.6477,85.4729)--(-73.0026,-20.1774,65.2955)--(-73.0026,20.1774,65.2955)--(-40.3548,32.6477,85.4729)--cycle
	(-20.1774,0,-97.9432)--(-40.3548,32.6477,-85.4729)--(-73.0026,20.1774,-65.2955)--(-73.0026,-20.1774,-65.2955)--(-40.3548,-32.6477,-85.4729)--cycle
	(97.9432,20.1774,0)--(85.4729,40.3548,32.6477)--(65.2955,73.0026,20.1774)--(65.2955,73.0026,-20.1774)--(85.4729,40.3548,-32.6477)--cycle
	(0,97.9432,20.1774)--(32.6477,85.4729,40.3548)--(20.1774,65.2955,73.0026)--(-20.1774,65.2955,73.0026)--(-32.6477,85.4729,40.3548)--cycle
	(-97.9432,20.1774,0)--(-85.4729,40.3548,-32.6477)--(-65.2955,73.0026,-20.1774)--(-65.2955,73.0026,20.1774)--(-85.4729,40.3548,32.6477)--cycle
	(0,97.9432,-20.1774)--(-32.6477,85.4729,-40.3548)--(-20.1774,65.2955,-73.0026)--(20.1774,65.2955,-73.0026)--(32.6477,85.4729,-40.3548)--cycle
	(-97.9432,-20.1774,0)--(-85.4729,-40.3548,32.6477)--(-65.2955,-73.0026,20.1774)--(-65.2955,-73.0026,-20.1774)--(-85.4729,-40.3548,-32.6477)--cycle
	(0,-97.9432,-20.1774)--(32.6477,-85.4729,-40.3548)--(20.1774,-65.2955,-73.0026)--(-20.1774,-65.2955,-73.0026)--(-32.6477,-85.4729,-40.3548)--cycle
	(97.9432,-20.1774,0)--(85.4729,-40.3548,-32.6477)--(65.2955,-73.0026,-20.1774)--(65.2955,-73.0026,20.1774)--(85.4729,-40.3548,32.6477)--cycle
	(0,-97.9432,20.1774)--(-32.6477,-85.4729,40.3548)--(-20.1774,-65.2955,73.0026)--(20.1774,-65.2955,73.0026)--(32.6477,-85.4729,40.3548)--cycle
	(85.4729,40.3548,-32.6477)--(97.9432,20.1774,0)--(97.9432,-20.1774,0)--(85.4729,-40.3548,-32.6477)--(73.0026,-20.1774,-65.2955)--(73.0026,20.1774,-65.2955)--cycle
	(85.4729,-40.3548,32.6477)--(97.9432,-20.1774,0)--(97.9432,20.1774,0)--(85.4729,40.3548,32.6477)--(73.0026,20.1774,65.2955)--(73.0026,-20.1774,65.2955)--cycle
	(-85.4729,40.3548,32.6477)--(-97.9432,20.1774,0)--(-97.9432,-20.1774,0)--(-85.4729,-40.3548,32.6477)--(-73.0026,-20.1774,65.2955)--(-73.0026,20.1774,65.2955)--cycle
	(-85.4729,-40.3548,-32.6477)--(-97.9432,-20.1774,0)--(-97.9432,20.1774,0)--(-85.4729,40.3548,-32.6477)--(-73.0026,20.1774,-65.2955)--(-73.0026,-20.1774,-65.2955)--cycle
	(40.3548,32.6477,-85.4729)--(20.1774,0,-97.9432)--(-20.1774,0,-97.9432)--(-40.3548,32.6477,-85.4729)--(-20.1774,65.2955,-73.0026)--(20.1774,65.2955,-73.0026)--cycle
	(-40.3548,-32.6477,-85.4729)--(-20.1774,0,-97.9432)--(20.1774,0,-97.9432)--(40.3548,-32.6477,-85.4729)--(20.1774,-65.2955,-73.0026)--(-20.1774,-65.2955,-73.0026)--cycle
	(40.3548,-32.6477,85.4729)--(20.1774,0,97.9432)--(-20.1774,0,97.9432)--(-40.3548,-32.6477,85.4729)--(-20.1774,-65.2955,73.0026)--(20.1774,-65.2955,73.0026)--cycle
	(-40.3548,32.6477,85.4729)--(-20.1774,0,97.9432)--(20.1774,0,97.9432)--(40.3548,32.6477,85.4729)--(20.1774,65.2955,73.0026)--(-20.1774,65.2955,73.0026)--cycle
	(32.6477,85.4729,-40.3548)--(0,97.9432,-20.1774)--(0,97.9432,20.1774)--(32.6477,85.4729,40.3548)--(65.2955,73.0026,20.1774)--(65.2955,73.0026,-20.1774)--cycle
	(-32.6477,85.4729,40.3548)--(0,97.9432,20.1774)--(0,97.9432,-20.1774)--(-32.6477,85.4729,-40.3548)--(-65.2955,73.0026,-20.1774)--(-65.2955,73.0026,20.1774)--cycle
	(-32.6477,-85.4729,-40.3548)--(0,-97.9432,-20.1774)--(0,-97.9432,20.1774)--(-32.6477,-85.4729,40.3548)--(-65.2955,-73.0026,20.1774)--(-65.2955,-73.0026,-20.1774)--cycle
	(32.6477,-85.4729,40.3548)--(0,-97.9432,20.1774)--(0,-97.9432,-20.1774)--(32.6477,-85.4729,-40.3548)--(65.2955,-73.0026,-20.1774)--(65.2955,-73.0026,20.1774)--cycle
	(20.1774,65.2955,-73.0026)--(32.6477,85.4729,-40.3548)--(65.2955,73.0026,-20.1774)--(85.4729,40.3548,-32.6477)--(73.0026,20.1774,-65.2955)--(40.3548,32.6477,-85.4729)--cycle
	(85.4729,40.3548,32.6477)--(73.0026,20.1774,65.2955)--(40.3548,32.6477,85.4729)--(20.1774,65.2955,73.0026)--(32.6477,85.4729,40.3548)--(65.2955,73.0026,20.1774)--cycle
	(-73.0026,20.1774,65.2955)--(-40.3548,32.6477,85.4729)--(-20.1774,65.2955,73.0026)--(-32.6477,85.4729,40.3548)--(-65.2955,73.0026,20.1774)--(-85.4729,40.3548,32.6477)--cycle
	(-85.4729,40.3548,-32.6477)--(-65.2955,73.0026,-20.1774)--(-32.6477,85.4729,-40.3548)--(-20.1774,65.2955,-73.0026)--(-40.3548,32.6477,-85.4729)--(-73.0026,20.1774,-65.2955)--cycle
	(-85.4729,-40.3548,32.6477)--(-65.2955,-73.0026,20.1774)--(-32.6477,-85.4729,40.3548)--(-20.1774,-65.2955,73.0026)--(-40.3548,-32.6477,85.4729)--(-73.0026,-20.1774,65.2955)--cycle
	(-20.1774,-65.2955,-73.0026)--(-32.6477,-85.4729,-40.3548)--(-65.2955,-73.0026,-20.1774)--(-85.4729,-40.3548,-32.6477)--(-73.0026,-20.1774,-65.2955)--(-40.3548,-32.6477,-85.4729)--cycle
	(85.4729,-40.3548,-32.6477)--(65.2955,-73.0026,-20.1774)--(32.6477,-85.4729,-40.3548)--(20.1774,-65.2955,-73.0026)--(40.3548,-32.6477,-85.4729)--(73.0026,-20.1774,-65.2955)--cycle
	(20.1774,-65.2955,73.0026)--(32.6477,-85.4729,40.3548)--(65.2955,-73.0026,20.1774)--(85.4729,-40.3548,32.6477)--(73.0026,-20.1774,65.2955)--(40.3548,-32.6477,85.4729)--cycle;
\end{scope}


\pgfmathtruncatemacro{\tubelength}{20} % in "double hexagon lengths"
\pgfmathtruncatemacro{\tubecirumferenceatoms}{20} 
\pgfmathsetmacro{\tuberadius}{4}
\newcommand{\carboncolor}{green!80!black}
\newcommand{\bondcolor}{black}
\pgfmathsetmacro{\initialrotationangle}{270}
\pgfmathsetmacro{\atombondlengthfraction}{0.2}
\newcommand{\bonddrawoptions}{thin}

\begin{scope}[x={(-0.2cm,-0.5cm)}, y={(1cm,0cm)}, z={(0cm,1cm)},rotate=90,xshift=22cm,yshift=-10cm,scale=1.2]
\foreach \x in {1,...,\tubelength}
{ \pgfmathsetmacro{\bondlength}{2*3.14159265*\tuberadius/sqrt(3)/\tubecirumferenceatoms}
	\pgfmathsetmacro{\atomradius}{\bondlength*\atombondlengthfraction}
	\foreach \y in {1,...,\tubecirumferenceatoms}
	{   \pgfmathsetmacro{\rotangle}{\y/\tubecirumferenceatoms*360}
		\pgfmathsetmacro{\ycoord}{cos(\rotangle)*\tuberadius}
		\pgfmathsetmacro{\zcoord}{sin(\rotangle)*\tuberadius}
		\pgfmathtruncatemacro{\shadingcolor}{50*cos(\rotangle+\initialrotationangle)+50}
		\draw[\bonddrawoptions,\bondcolor!\shadingcolor!gray] (\x*3*\bondlength+0.5*\bondlength,\ycoord,\zcoord) -- (\x*3*\bondlength+1.5*\bondlength,\ycoord,\zcoord);
		\shade[ball color=\carboncolor!\shadingcolor] (\x*3*\bondlength+0.5*\bondlength,\ycoord,\zcoord) circle (\atomradius*1cm) ;
	}
	
	\foreach \y in {1,...,\tubecirumferenceatoms}
	{
		\pgfmathsetmacro{\ycoord}{cos(\y/\tubecirumferenceatoms*360)*\tuberadius}
		\pgfmathsetmacro{\zcoord}{sin(\y/\tubecirumferenceatoms*360)*\tuberadius}
		\pgfmathsetmacro{\rotangle}{\y/\tubecirumferenceatoms*360+360/2/\tubecirumferenceatoms}
		\pgfmathsetmacro{\ycoordtwo}{cos(\rotangle)*\tuberadius}
		\pgfmathsetmacro{\zcoordtwo}{sin(\rotangle)*\tuberadius}
		\pgfmathsetmacro{\rotanglethree}{\y/\tubecirumferenceatoms*360-360/2/\tubecirumferenceatoms}
		\pgfmathsetmacro{\ycoordthree}{cos(\rotanglethree)*\tuberadius}
		\pgfmathsetmacro{\zcoordthree}{sin(\rotanglethree)*\tuberadius}
		\pgfmathtruncatemacro{\shadingcolor}{50*cos(\rotangle+\initialrotationangle)+50}
		\draw[\bonddrawoptions,\bondcolor!\shadingcolor!gray] (\x*3*\bondlength+1.5*\bondlength,\ycoord,\zcoord) -- (\x*3*\bondlength+2*\bondlength,\ycoordtwo,\zcoordtwo);
		\draw[\bonddrawoptions,\bondcolor!\shadingcolor!gray] (\x*3*\bondlength+1.5*\bondlength,\ycoord,\zcoord) -- (\x*3*\bondlength+2*\bondlength,\ycoordthree,\zcoordthree);
		\shade[ball color=\carboncolor!\shadingcolor] (\x*3*\bondlength+1.5*\bondlength,\ycoord,\zcoord) circle (\atomradius*1cm);
	}
	
	\foreach \y in {1,...,\tubecirumferenceatoms}
	{   \pgfmathsetmacro{\rotangle}{\y/\tubecirumferenceatoms*360+360/2/\tubecirumferenceatoms}
		\pgfmathsetmacro{\ycoord}{cos(\rotangle)*\tuberadius}
		\pgfmathsetmacro{\zcoord}{sin(\rotangle)*\tuberadius}
		\pgfmathtruncatemacro{\shadingcolor}{50*cos(\rotangle+\initialrotationangle)+50}
		\draw[\bonddrawoptions,\bondcolor!\shadingcolor!gray] (\x*3*\bondlength+2*\bondlength,\ycoord,\zcoord) -- (\x*3*\bondlength+3*\bondlength,\ycoord,\zcoord);
		\shade[ball color=\carboncolor!\shadingcolor] (\x*3*\bondlength+2*\bondlength,\ycoord,\zcoord) circle (\atomradius*1cm);
	}
	
	\foreach \y in {1,...,\tubecirumferenceatoms}
	{   \pgfmathsetmacro{\ycoord}{cos(\y/\tubecirumferenceatoms*360)*\tuberadius}
		\pgfmathsetmacro{\zcoord}{sin(\y/\tubecirumferenceatoms*360)*\tuberadius}
		\pgfmathsetmacro{\rotangle}{\y/\tubecirumferenceatoms*360+360/2/\tubecirumferenceatoms}
		\pgfmathsetmacro{\ycoordtwo}{cos(\rotangle)*\tuberadius}
		\pgfmathsetmacro{\zcoordtwo}{sin(\rotangle)*\tuberadius}
		\pgfmathsetmacro{\rotanglethree}{\y/\tubecirumferenceatoms*360-360/2/\tubecirumferenceatoms}
		\pgfmathsetmacro{\ycoordthree}{cos(\rotanglethree)*\tuberadius}
		\pgfmathsetmacro{\zcoordthree}{sin(\rotanglethree)*\tuberadius}
		\pgfmathtruncatemacro{\shadingcolor}{50*cos(\rotangle+\initialrotationangle)+50}
		\draw[\bonddrawoptions,\bondcolor!\shadingcolor!gray] (\x*3*\bondlength+3.5*\bondlength,\ycoord,\zcoord) -- (\x*3*\bondlength+3*\bondlength,\ycoordtwo,\zcoordtwo);
		\draw[\bonddrawoptions,\bondcolor!\shadingcolor!gray] (\x*3*\bondlength+3.5*\bondlength,\ycoord,\zcoord) -- (\x*3*\bondlength+3*\bondlength,\ycoordthree,\zcoordthree);
	}
	\foreach \y in {1,...,\tubecirumferenceatoms}
	{ \pgfmathsetmacro{\rotangle}{\y/\tubecirumferenceatoms*360+360/2/\tubecirumferenceatoms}
		\pgfmathsetmacro{\ycoordtwo}{cos(\rotangle)*\tuberadius}
		\pgfmathsetmacro{\zcoordtwo}{sin(\rotangle)*\tuberadius}
		\pgfmathtruncatemacro{\shadingcolor}{50*cos(\rotangle+\initialrotationangle)+50}
		\shade[ball color=\carboncolor!\shadingcolor] (\x*3*\bondlength+3*\bondlength,\ycoordtwo,\zcoordtwo) circle (\atomradius*1cm);
	}
	
}

\end{scope}






\begin{scope}[xshift=30cm,yshift=35cm,scale=2]
% some styles
\tikzset{
every node/.style={anchor=west,
	regular polygon, 
	regular polygon sides=6,
	draw,
	line width=5pt,
	minimum width=5cm,
	outer sep=0,
	}
}
\node (A) {};
\node (B) at (A.corner 1) {B};
\node (C) at (B.corner 1) {C};
\node (D) at (A.corner 5) {D};
\node (E) at (B.corner 5) {};	
\node (F) at (C.corner 5) {};
\node (G) at (D.corner 5) {};
\node (H) at (E.corner 5) {};
\node (I) at (F.corner 5) {};
\node (J) at (G.corner 5) {};
\node (K) at (H.corner 5) {};
\node (L) at (I.corner 5) {};
\node (M) at (J.corner 5) {};
\node (N) at (K.corner 5) {};
\node (O) at (L.corner 5) {};
\node (P) at (M.corner 5) {};
\node (Q) at (N.corner 5) {};
\node (R) at (O.corner 5) {};
\foreach \hex in {A,...,R}
{
	\foreach \corn in {1,...,6}
	\draw[ball color=blue!90!white,line width=0.1pt, color=green!90!white] (\hex.corner \corn) circle (7pt); 
}

\end{scope}


\begin{scope}[xshift=78cm,yshift=48cm,scale=5]
%%%  define vertices with coordinates
\coordinate (0;0) at (0,0); 
\foreach \c in {1,...,3}{%  
	\foreach \i in {0,...,5}{% 
		\pgfmathtruncatemacro\j{\c*\i}
		\coordinate (\c;\j) at (60*\i:\c);  
} }
\foreach \i in {0,2,...,10}{% 
	\pgfmathtruncatemacro\j{mod(\i+2,12)} 
	\pgfmathtruncatemacro\k{\i+1}
	\coordinate (2;\k) at ($(2;\i)!.5!(2;\j)$) ;}

\foreach \i in {0,3,...,15}{% 
	\pgfmathtruncatemacro\j{mod(\i+3,18)} 
	\pgfmathtruncatemacro\k{\i+1} 
	\pgfmathtruncatemacro\l{\i+2}
	\coordinate (3;\k) at ($(3;\i)!1/3!(3;\j)$)  ;
	\coordinate (3;\l) at ($(3;\i)!2/3!(3;\j)$)  ;
}

%%%%%%%%% draw lines %%%%%%%%
\foreach \i in {0,...,6}{% 
	\pgfmathtruncatemacro\k{\i}
	\pgfmathtruncatemacro\l{15-\i}
	\draw[line width=3pt,black,loosely dashed] (3;\k)--(3;\l);
	\pgfmathtruncatemacro\k{9-\i} 
	\pgfmathtruncatemacro\l{mod(12+\i,18)}   
	\draw[line width=3pt,black,loosely dashed] (3;\k)--(3;\l); 
	\pgfmathtruncatemacro\k{12-\i} 
	\pgfmathtruncatemacro\l{mod(15+\i,18)}   
	\draw[line width=3pt,black,loosely dashed] (3;\k)--(3;\l);}    
%%%%%%%%% draw points %%%%%%%% 
\node[regular polygon, regular polygon sides=6,draw,
minimum width=2.5cm,
outer sep=0,very thick,green,fill=white] at (0;0) {};
\foreach \c in {1,...,3}{%
	\pgfmathtruncatemacro\k{\c*6-1}    
	\foreach \i in {0,...,\k}{% 
		\node[regular polygon, regular polygon sides=6,draw,
		minimum width=2.5cm	,outer sep=0,very thick,green,fill=white] at (\c;\i) {};
}}  
 
\end{scope}

\begin{scope}
\draw[->, >=latex,fill=orange, line width=40pt,rotate=-68,scale=7.79, color=orange, opacity=0.2]
(0, 0) -- node [black,sloped] {} +(100:10cm);
\end{scope}

\node[text width=2cm,align=center,scale=7]  at (0,-13) { Fullerene\\1985 };
\node[text width=4cm,align=center,scale=7]  at (30,3) {Carbon Nanotube\\1991 };
\node[text width=2cm,align=center,scale=7]  at (55,18) {Graphene\\2004 };
\node[text width=4cm,align=center,scale=7]  at (78,28) {graphene allotropes\\ undiscovered  \\ for example  $sp-sp^{2}$\\20?? };
\end{tikzpicture}




    
\end{document}