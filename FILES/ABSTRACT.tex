\cleardoublepage
%\vspace{-10cm}

%	{\Large\textbf{Crecimiento y caracterización de microcavidades ópticas}}
%	\end{center}
%	{\large\textbf{Resumen:}\\}
%	En esta tesis se muestran los resultados obtenidos en el crecimiento y caracterización de microcavidades ópticas. En la primera parte de esta  tesis se muestran los aspectos fundamentales de estas estructuras, así como los fenómenos de la interacción luz-materia que se pueden llevar a cabo en \'estas. En la segunda parte se describe detalladamente el diseño y la composición de las microcavidades crecidas. Adem\'as se muestran y se describen  los arreglos  experimentales utilizados para la caracterización óptica.  Finalmente se presentan los resultados de los crecimientos realizados  y la caracterización de los mismos.
%	
\chapter{Abstract}%
In this work a structural study of exfoliated graphene layers on a SiO2/Si substrate and of graphene nanoribbons (GNRs) grown on SiC (0 0 0 1) substrates is addressed. On the one hand, the high quality mechanically exfoliated graphene layers present a distribution of different thicknesses so it is essential to characterize the topography of the layers in the submicrometric regime. Regarding GNRs these turn out to be structures with interesting optical and electronic properties with potential applications in the field of optoelectronics and nanoelectronics. In both cases it is essential to develop fast and non-destructive approaches to determine the thickness and uniformity properties of the studied nanostructures.  That is why the differential contrast technique (DRC) based on a scanning near-field optical microscope (NSOM) is implemented to achieve these resolutions, this technique consists of taking the numerical difference between the reflectivity coming from a region with substrate and a region containing graphene. In the case of monolayers we rely on the ellipsometry technique and for both systems on a multiple reflection model (graphene/$SiO_2$/Si and graphene/SiC system, respectively) to know that the optical contrast can be modulated by changing the thickness of the $SiO_2$ or SiC layer and the incident wavelength.  

With this approach, it is possible to evaluate GNRs widths with dimensions as small as 60 nm with a thickness of one or two graphene monolayers and with a spatial resolution of 40 nm. Our results demonstrate that the DRC technique is powerful for analyzing the morphology of GNRs grown on SiC (0 0 0 1) substrates as well as exfoliated monolayers of graphene on SiO2/ which prove to be a promising wafer-scale platform for the development of graphene-based nanoelectronics.

In addition, the optical characterization of CdTe based materials as well as the response to evaporation of Ag thin films is shown as an approach to understand the behavior of topological insulators and the idea lies in how thin layers of metals are seen on semiconductors and their monitoring. Another group of materials as a collaboration with ETH that were studied were CuSn on polymers of different geometry and finishing processes for their synthesis. The objective is to be able to see the stress in a macro way and at the grain boundaries so $RDS$ and macro $RDS$ were used. \\
Some contributions were made with the industry, where in very general terms we will show what was done to see the potential of the techniques used in different materials and analysis interests. 

%	\noindent\rule[2pt]{\textwidth}{0.8pt}




%\cleardoublepage
%