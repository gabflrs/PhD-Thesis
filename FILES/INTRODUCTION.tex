\chapter{Introduction}
\label{chapter:Introduction}
\textit{This chapter includes the motivation to carry out this doctoral work as well as the purposes of the same, on other hand, it describes the order of the thesis in which the description of our central material which is graphene, passing through additional projects until the interpretation of the experimental results obtained and conclusions are dealt with.}
\vfill
\minitoc\newpage


\section{Aims and Objectives}
\vspace{-1cm}
\lettrine[lines=3, lraise=0.1, nindent=0mm, slope=0mm]{\textbf{B}}{}oth graphene and graphene nanoribbons turn out to be quite interesting systems from their structure to the promising applications, the fact that we research terms of their optical and morphological properties opens the panorama for these to be investigated in-depth and to contribute to the implementation of new technologies based on this material. \\

The work carried out has the purpose of knowing both the optical and structural properties of two-dimensional and one-dimensional materials, we show the potential of optical techniques from the non-invasive point of view as well as the ability to study at the necessary scale of these systems for which we rely on various techniques such as Ellipsometry, DRC (differential reflectance contrast) based on NSOM (Near-field scanning optical microscope), RAMAN and AFM, the studies performed show i) that the DRC technique is powerful to analyze the morphology of GNRs grown on \textit{SiC} substrates (0001) and is quite promising for the development of nanoelectronics based on graphene, as well as the characterization of low scale systems such as dichalcogenides, ii) that RDS is very useful to study \textit{CuSn} on polymer even though it is a non-homogeneous material and it is still possible to rescue information related to stress in the structure, iii) The potential of SE and RDS to monitor the surface modification due to \textit{Ag} evaporation on \textit{CdTe} surfaces. 

\section{Thesis Outline}
\vspace{-1cm}
\lettrine[lines=3, lraise=0.1, nindent=0mm, slope=0mm]{\textbf{T}}{}he content of this thesis includes 5 chapters and 2 appendices. The first chapter details the motivation to study graphene in the first instance and why it is important and challenging to perform studies for structural characterization, it also details the importance of systems based on \textit{CdTe} as well as \textit{CuSn}. The second chapter details the background knowledge and the state of the art of graphene, from a fundamental description to the behavior of phonons in these structures (graphene monolayers, bilayers, and nanoribbons), as well as it is necessary to highlight the electrical properties, the importance of the characterizations is addressed since the main challenge is the scalability of these materials for mass reproduction and implementation in applications.  In chapter 3 we describe the characterization techniques used from the physical fundamentals to the description of the experimental setup for the acquisition of the information by which we describe NSOM, Raman, DRC, RDS. Within this chapter, we also describe the samples studied as part of this synthesis and finally how is the complete structure. Chapter 4 deals with the experimental results for each sample studied in the techniques used, the thesis work dealt mostly with samples of monolayer and bilayer GNRs, here we will detail how it is possible to have the ability to discern these systems optically and the contribution of the research carried out. Finally, the last chapter reports the main results of the thesis,  describes the discussion of the results obtained and the perspectives of the work performed. As appendixes, two different systems that were also studied as part A are those based on \textit{CdTe }as an input to know the behavior of topological insulators which included working in a UHV environment as well as electron beam evaporation and a mass spectrometer. Which will be detailed in this section, as well as a set of samples based on \textit{CuSn} which have a range of applications in the field of metallurgy and tracks for circuits, it is worth mentioning that the thesis work focused on the primary studies for future industrial uses and applications in various technological areas. 
